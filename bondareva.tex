\documentclass[10pt,a4paper,titlepage]{article}
\usepackage[utf8]{inputenc}
\usepackage{amsmath}
\usepackage{amsfonts}
\usepackage{amssymb}
\usepackage{amsthm}
\author{Felix Fritz}
\title{The Bondareva-Shapley Theorem}

\theoremstyle{plain}
\newtheorem{thm}{Theorem}[section] % reset theorem numbering for each chapter

\theoremstyle{definition}
\newtheorem{defn}[thm]{Definition} % definition numbers are dependent on theorem numbers
\newtheorem{exmp}[thm]{Example} % same for example numbers

\begin{document}
\maketitle

\tableofcontents
\pagebreak

\section{Introduction}
 Before diving into the concept of balanced collections, balanced games and the Bondareva-Shapley Theorem, I want to establish a basic understanding of cooperative games, imputations and the core, mainly as it is described in Robert P. Gilles book \cite{gilles}, pp. 12--13, 18--20 and 29--35.

 \subsection{Cooperative Games and the Core}
 First we introduce a formal definition of a cooperative game.
\begin{defn}
    The pair $(N, v)$ is a \textit{cooperative game} if $N$ is a finite player set and $v: 2^N \rightarrow \mathbb{R}$ is a characteristic function that assigns to every coalition $S \subset N$ an attainable payoff $v(S)$ such that $v(\emptyset) = 0$.
    
    For every player set $N$ we denote by $\mathcal{G}^N$ the class of all characteristic functions on $N$.
\end{defn}
Simply put, in a cooperative game we define collective payoff values for every possible coalition between players. A coalition without any players leads to a payoff of zero.

Suppose for a given cooperative game we form a grand coalition, meaning all player work together for a total payoff of $v(N)$. 



 \section{Balanced Games and the Bondareva-Shapley Theorem}
 In the following chapter we will discuss the properties of cooperative games with a non-empty core and introduce the notion of balancedness, which leads us to the Bondareva-Shapley Theorem.

 \section{Market Games with nonempty cores}
 Market games can be proven to have a non-empty core using the Bondareva-Shapley Theorem. We will take a closer look on how this can be achieved in the third chapter.

 \section{Proving the Theorem using Linear Programming}
 Finally, we will show that the Bondareva-Theorem holds true with a prove using the Duality Problem in Linear Programming.
 
\pagebreak
 
\bibliographystyle{unsrt}
\bibliography{references}
\end{document}